% Chapter Template

\chapter{Audio Decoding Algorithm Design} % Main chapter title

\label{Chapter5} % Change X to a consecutive number; for referencing this chapter elsewhere, use \ref{ChapterX}


\section{Synthetic Audio Generation}

Prior to investing significant resources in PCB design, we found it prudent to develop a synthetic model of the ideal audio output of the DIY Supercube. This model consists of a synthetic audio generator that, when given a specific move sequence, generates a `.wav` file consisting of the distinct time series tones that the DIY Supercube would theoretically produce. The synthetic data produced by this model then served as the first test cases for the final audio decoding model.

\subsection{Representing the Audio Protocol}
TODO pull from Jupyter notebook

\subsection{Representing the Rubik's Cube}
TODO pull from Jupyter Notebook

\subsection{Generating the Audio for an Arbitary Algorithm}
TODO pull from Jupyter Notebook

TODO show a spectrogram of the generated audio here


\section{Decoding the Synthetic Audio}

TODO pull from Jupyter Notebook

\subsection{Examining the Waveform}
TODO pull from Jupyter Notebook

\subsection{Conversion to Strength of Individual Frequencies}
TODO pull from Jupyter Notebook


\section{Adding Realisitic Noise to the Synthetic Audio}

TODO pull from Jupyter Notebook


\section{Decoding the Noisy Synthetic Audio}

TODO pull from Jupyter Notebook

\subsection{Optimizing algorithm parameters}
TODO - Share the strategy for finding the optimal parameters, and the end results, but defer the detailed analysis for the Evaluation.
