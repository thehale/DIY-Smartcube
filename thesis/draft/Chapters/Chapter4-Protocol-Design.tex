% Chapter Template

\chapter{Protocol Design} % Main chapter title

\label{Chapter4} % Change X to a consecutive number; for referencing this chapter elsewhere, use \ref{ChapterX}

\section{Introduction}
Designing a sound-based protocol for tracking the moves of a Rubik's Cube requires great care.
Lots of things produce sound that could interfere with the protocol: people talking, machines operating, nature stirring, and so forth.
Furthermore, the structure of the Rubik's Cube itself imposes stringent physical constraints on the size of any components used to produce the sounds used in the protocol.

This chapter first seeks to clearly detail the specific constraints that must be considered when designing a sound-based protocol for tracking the moves of a Rubik's Cube.
From there, a specific protocol will be proposed for use in the successive chapters of this thesis.


\section{Requirements}


\subsection{Signal to Noise Ratio}
Sound is an easily accessible, and therefore noisy, medium of communication.
As a result, any data transmission protocol based on sound must be resilient to the presence of additional noise unrelated to the signal being transmitted.
For a sound-based protocol to be effective, the tones used in the protocol must be easily distinguishable from background noise.

For the purpose of this thesis, "background noise" will be considered the ambient noise present in a quiet room when a speedcuber is actively solving a Rubik's Cube.
To measure this background noise, three recordings were taken in a household bedroom (a typical place for a speedcuber to practice solving the cube) while solving a variety of speedcubes selected based on their "noisyness" along with a fourth control recording taken of just the ambient noise in the bedroom (i.e. while no cubes were being solved).
The recordings were then analyzed to see which frequencies were most prevalent in the recorded audio. 
% TODO re-record the audio without the actual tones of the test turn sequence. Make a table and/or multi-part graphic of the resulting data. Draw conclusions of a quantitative definition of "background noise"
\begin{enumerate}
    \item Background noise in a room without any use of a speedcube (Control).
    \item Background noise in a room while solving a "quiet" speedcube (the Gans 356).
    \item Background noise in a room while solving a "standard-volume" speedcube (the Gans XS).
    \item Background noise in a room while solving a "noisy" speedcube (the QiYi Qimeng).
\end{enumerate}
TODO incorporate these pieces into the above paragraph.
\begin{itemize}
    \item Lower frequency tones (e.g. < 100Hz) are more prominent in background noise.
    \item Analysis based on this criterion should also consider the background noise generated by the rotation of the Rubik's Cube itself.
    \item Most of the concerns involved in this criteria are solved by increasing the volume of the generated tone.
    \item Contrast with the WCA regulation. Perhaps a protocol that is weaker in noisy environments is more likely to be permissible.
\end{itemize}

\subsection{Tone Distinctiveness}
Next, if more than one tone is required for the protocol, each tone must be unique enough to be easily distinguished from each other tone.
Since a standard smartphone or laptop microphone will be used on the listening end of this protocol, the definition of "easily distinguishable" must be based on an assessment of how clearly smartphone/laptop grade microphones can distinguish similar frequencies. % TODO continue this discussion.

Some sound-based communcation protocols (e.g. Morse Code) can communicate through a single frequency by controlling the duration/pattern of activation of that frequency. \cite{TODO}
\begin{itemize}
    \item Our testing with smartphone microphones suggested that tones needed to be separated by at least 100Hz to be uniquely detected from each other.
    \item We can also exploit the fact that only one of each faces four positions will be active at a time.
\end{itemize}

\subsection{Ease of Deployment on Consumer Hardware}
Each selected tone must be easily produced by a consumer grade speaker and consumed by a consumer grade microphone.
\begin{itemize}
    \item A typical smartphone can produce and consume tones in the human auditory range of 10Hz - 20kHz \cite{TODO}.
\end{itemize}

\subsection{Human Auditory Range}
Tone selection shall be biased in favor of tones beyond the typical human auditory range; however, in intial prototypes, this criterion is subordinate to all other criteria.
\begin{itemize}
    \item Many humans are unable to hear tones above 17kHz (TODO how many? source?)
\end{itemize}


\section{Absolute Sound Positioning}

Each of the Rubik's Cube's six faces has four possible positions, for a total of 24 unique tones required for our protocol.

TODO finish this protocol proposal.


\section{Alternative Protocol Designs}

\subsection{Relative Sound Positioning}

This strategy seeks to minimize the number of discrete frequencies required to communicate changes in the cube's state. Since the cube consists of 6 faces, and each face can be turned either clockwise or counterclockwise, one could design a two-tone protocol using only 8 discrete audio frequencies to build the smart cube. The first tone would come from one of six predefined audio bands, one for each face of the cube. The second tone would come from one of two separately predefined audio bands, one for each possible direction of rotation. From this, an audio processing model could be designed to process a sequence of these two-tone pairs and reconstruct the sequence of face rotations by recording the rotated face followed by its direction of rotation.

However this model presents challenges. Take for example, a speedcuber averaging 5 turns per second (common for a 12-15 second solver) with bursts up to 10 TPS. The burst TPS would require the successful transmission of 20 sequential tones within a single second - only 50ms per tone, all in the midst of additional noise from the cube's pieces hitting each other harder at the higher turn speed. And, to cap it all off, since each tone is only ever transmitted once, the audio detection model must achieve 100\% tone recognition to be able to accurately reconstruct the originating move sequence. As a result, this model fails to support any sort of error correction that would make it resistant to the common challenges to data transmission through sound.

However this model inherently precludes robust error checking procedures. Each tone that is not accurately detectedany of the tones are not accurately detected, the data of that particular rotation cannotit becomes impossible to entirely reconstruct the executed move sequence

TODO finish discussing this protocol proposal


