% Chapter Template

\chapter{Protocol Design} % Main chapter title

\label{Chapter4} % Change X to a consecutive number; for referencing this chapter elsewhere, use \ref{ChapterX}


\section{Requirements}

Sound is an easily accessible, and therefore noisy medium of communication. As a result, any data transmission protocol must be resilient to the presence of additional noise.

\subsection{Signal to Noise Ratio}
Each selected tone must be easily distinguished from background noise in a quiet room.
\begin{itemize}
    \item Lower frequency tones (e.g. < 100Hz) are more prominent in background noise.
    \item Analysis based on this criterion should also consider the background noise generated by the rotation of the Rubik's Cube itself.
    \item Most of the concerns involved in this criteria are solved by increasing the volume of the generated tone.
    \item Contrast with the WCA regulation. Perhaps a protocol that is weaker in noisy environments is more likely to be permissible.
\end{itemize}

\subsection{Tone Distinctiveness}
Each selected tone must be easily distinguished from each other tone. 
\begin{itemize}
    \item Our testing with smartphone microphones suggested that tones needed to be separated by at least 100Hz to be uniquely detected from each other.
    \item We can also exploit the fact that only one of each faces four positions will be active at a time.
\end{itemize}

\subsection{Ease of Deployment on Consumer Hardware}
Each selected tone must be easily produced by a consumer grade speaker and consumed by a consumer grade microphone.
\begin{itemize}
    \item A typical smartphone can produce and consume tones in the human auditory range of 10Hz - 20kHz (TODO source).
\end{itemize}

\subsection{Human Auditory Range}
Tone selection shall be biased in favor of tones beyond the typical human auditory range; however, in intial prototypes, this criterion is subordinate to all other criteria.
\begin{itemize}
    \item Many humans are unable to hear tones above 17kHz (TODO how many? source?)
\end{itemize}


\section{Absolute Sound Positioning}

Each of the Rubik's Cube's six faces has four possible positions, for a total of 24 unique tones required for our protocol.

Sed ullamcorper quam eu nisl interdum at interdum enim egestas. Aliquam placerat justo sed lectus lobortis ut porta nisl porttitor. Vestibulum mi dolor, lacinia molestie gravida at, tempus vitae ligula. Donec eget quam sapien, in viverra eros. Donec pellentesque justo a massa fringilla non vestibulum metus vestibulum. Vestibulum in orci quis felis tempor lacinia. Vivamus ornare ultrices facilisis. Ut hendrerit volutpat vulputate. Morbi condimentum venenatis augue, id porta ipsum vulputate in. Curabitur luctus tempus justo. Vestibulum risus lectus, adipiscing nec condimentum quis, condimentum nec nisl. Aliquam dictum sagittis velit sed iaculis. Morbi tristique augue sit amet nulla pulvinar id facilisis ligula mollis. Nam elit libero, tincidunt ut aliquam at, molestie in quam. Aenean rhoncus vehicula hendrerit.


\section{Relative Sound Positioning}

This strategy seeks to minimize the number of discrete frequencies required to communicate changes in the cube's state. Since the cube consists of 6 faces, and each face can be turned either clockwise or counterclockwise, one could design a two-tone protocol using only 8 discrete audio frequencies to build the smart cube. The first tone would come from one of six predefined audio bands, one for each face of the cube. The second tone would come from one of two separately predefined audio bands, one for each possible direction of rotation. From this, an audio processing model could be designed to process a sequence of these two-tone pairs and reconstruct the sequence of face rotations by recording the rotated face followed by its direction of rotation.

However this model presents challenges. Take for example, a speedcuber averaging 5 turns per second (common for a 12-15 second solver) with bursts up to 10 TPS. The burst TPS would require the successful transmission of 20 sequential tones within a single second - only 50ms per tone, all in the midst of additional noise from the cube's pieces hitting each other harder at the higher turn speed. And, to cap it all off, since each tone is only ever transmitted once, the audio detection model must achieve 100\% tone recognition to be able to accurately reconstruct the originating move sequence. As a result, this model fails to support any sort of error correction that would make it resistant to the common challenges to data transmission through sound.

However this model inherently precludes robust error checking procedures. Each tone that is not accurately detectedany of the tones are not accurately detected, the data of that particular rotation cannotit becomes impossible to entirely reconstruct the executed move sequence


