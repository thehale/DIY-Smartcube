% Chapter Template

\chapter{Introduction} % Main chapter title

\label{Chapter1} % Change X to a consecutive number; for referencing this chapter elsewhere, use \ref{ChapterX}

% TODO What are we looking at? What will be shown? State general research questions to answer. Start by defining the *problem*. The rest of the thesis will detail my solution.

% I am trying to persuade someone that a sound-based smart cube is viable on consumer grade hardware.

\section{The Advent of Smart Cubes}

Speedsolving, the sport of solving twisty puzzles like the Rubik’s Cube
as fast as possible, has seen a resurgence of popularity since the
early 2000s. \cite{TODO} Over the past two decades many advances in
cube technology have produced ever higher performing puzzles.

Recently, the speedcubing community has seen the entrance of smart
cubes, special versions of a Rubik’s Cube built around hardware that
can connect to a mobile device over Bluetooth. These smart cubes have
sparked a wave of excitement with the vast opportunities they offer for
automatic turn tracking, performance analysis, personalized improvement
feedback, and networked competition.

\section{Obstacles to Adoption}

While a revolutionary idea, smart cubes still face several obstacles to
widespread adoption.

\begin{itemize}

    \item \emph{Cost}: Smart cubes can cost up to eight times as much
    as a comparable non-smart speedcube. \footnote{For example, one
    popular budget speedcube, the Moyu Weilong, costs only \$5, while
    the cheapest smartcube, the Giiker Cube, starts at \$40.
    \cite{TODO}. On the higher end, a premium speedcube, like the Gans
    356 XS, retails for just over \$60 while a premium smartcube, like
    the GoCube, retails for over \$100. \cite{TODO}}
    
    \item \emph{Performance}: Existing smart cubes turn slower than
    comparable non-smart cubes. \cite{TODO}
    
    \item \emph{Reliability}: Many smart cube owners report inability
    to connect the smart cube to a mobile device and missed/inaccurate
    turn tracking. \cite{TODO}
    
    \item \emph{Regulation}: Current competition rules ban the use of
    electronics during timed solves, thus banning the use of smart
    cubes. There is no foreseeable change to this rule. \cite{TODO} %
    WCA Regulations
    
\end{itemize}

As a result of these obstacles, many speedcubers refrain from
purchasing a smart cube, despite expressing significant interest in the
opportunities smart cubes offer.

Furthermore, all current smartcubes have been specifically built for
the primary purpose of providing move-tracking functionality. There is
no existing way to automatically track the moves of a standard,
"non-smart" speedcube.


\section{Purpose of this Thesis}

The primary goal of this thesis is to create a proof-of-concept for a
smart cube design that can enable a speedcuber to use his/her personal
favorite cube, while still having all the benefits of a smart cube.

In other words, this thesis seeks to answer the following question:

\emph{Is it possible to track the face turns of a standard, "non-smart"
speedcube in a non-destructive, competition-legal way?}

\section{Thesis Overview}

TODO give an overview of the rest of the Thesis document.